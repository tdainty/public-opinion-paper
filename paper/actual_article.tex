\documentclass[11pt,]{article}
\usepackage[left=1in,top=1in,right=1in,bottom=1in]{geometry}
\newcommand*{\authorfont}{\fontfamily{phv}\selectfont}
\usepackage{lmodern}


  \usepackage[T1]{fontenc}
  \usepackage[utf8]{inputenc}



\usepackage{abstract}
\renewcommand{\abstractname}{}    % clear the title
\renewcommand{\absnamepos}{empty} % originally center

\renewenvironment{abstract}
 {{%
    \setlength{\leftmargin}{0mm}
    \setlength{\rightmargin}{\leftmargin}%
  }%
  \relax}
 {\endlist}

\makeatletter
\def\@maketitle{%
  \newpage
%  \null
%  \vskip 2em%
%  \begin{center}%
  \let \footnote \thanks
    {\fontsize{18}{20}\selectfont\raggedright  \setlength{\parindent}{0pt} \@title \par}%
}
%\fi
\makeatother




\setcounter{secnumdepth}{0}

\usepackage{longtable,booktabs}
\usepackage{caption}
\captionsetup{font=bf}


\title{An Example Article\thanks{The paper's revision history and the
materials needed to reproduce its analyses can be found
\href{http://github.com/tdainty/example_article}{on Github here}.
Corresponding author:
\href{mailto:thomas-dainty@uiowa.edu}{\nolinkurl{thomas-dainty@uiowa.edu}}.
Current version: November 18, 2024.}  }
 



\author{\Large Thomas
Dainty\vspace{0.05in} \newline\normalsize\emph{University of Iowa}  }


\date{}

\usepackage{titlesec}

\titleformat*{\section}{\normalsize\bfseries}
\titleformat*{\subsection}{\normalsize\itshape}
\titleformat*{\subsubsection}{\normalsize\itshape}
\titleformat*{\paragraph}{\normalsize\itshape}
\titleformat*{\subparagraph}{\normalsize\itshape}

\newcommand{\dummy}[1]{#1}

\usepackage{natbib}
\bibpunct{(}{)}{;}{a}{}{,}
\bibliographystyle{apsr}
%\usepackage[strings]{underscore} % protect underscores in most circumstances



\newtheorem{hypothesis}{Hypothesis}
\usepackage{setspace}

\makeatletter
\@ifpackageloaded{hyperref}{}{%
\ifxetex
  \PassOptionsToPackage{hyphens}{url}\usepackage[setpagesize=false, % page size defined by xetex
              unicode=false, % unicode breaks when used with xetex
              xetex]{hyperref}
\else
  \PassOptionsToPackage{hyphens}{url}\usepackage[unicode=true]{hyperref}
\fi
}

\@ifpackageloaded{color}{
    \PassOptionsToPackage{usenames,dvipsnames}{color}
}{%
    \usepackage[usenames,dvipsnames]{color}
}
\makeatother
\hypersetup{breaklinks=true,
%            bookmarks=true,
            pdfauthor={Thomas Dainty (University of Iowa)},
             pdfkeywords = {these, always seem silly, to me, given
google, but regardless},  
            pdftitle={An Example Article},
            colorlinks=true,
            citecolor=black,
            urlcolor=blue,
            linkcolor=black,
            pdfborder={0 0 0}}
\urlstyle{same}  % don't use monospace font for urls

% set default figure placement to htbp
\makeatletter
\def\fps@figure{htbp}
\makeatother



% add tightlist ----------
\providecommand{\tightlist}{%
\setlength{\itemsep}{0pt}\setlength{\parskip}{0pt}}

\begin{document}
	
% \pagenumbering{arabic}% resets `page` counter to 1 
%    

% \maketitle

{% \usefont{T1}{pnc}{m}{n}
\setlength{\parindent}{0pt}
\thispagestyle{plain}
{\fontsize{18}{20}\selectfont\raggedright 
\maketitle  % title \par  

}

{
   \vskip 13.5pt\relax \normalsize\fontsize{11}{12} 
\textbf{\authorfont Thomas Dainty} \hskip 15pt \emph{\small University
of Iowa}   

}

}








\begin{abstract}

    \hbox{\vrule height .2pt width 39.14pc}

    \vskip 8.5pt % \small 

\noindent Here's where you write 100 to 250 words, depending on the
journal, that describe your objective, methods, results, and conclusion.


\vskip 8.5pt \noindent \emph{Keywords}: these, always seem silly, to me,
given google, but regardless \par

    \hbox{\vrule height .2pt width 39.14pc}



\end{abstract}


\vskip 6.5pt


\noindent \singlespacing \section{Introduction}\label{introduction}

This is a test.

\section{Literature Review}\label{literature-review}

\subsection{Topic 1}\label{topic-1}

\subsubsection{Topic 2}\label{topic-2}

\section{Theory}\label{theory}

\section{Research Design}\label{research-design}

\section{Results}\label{results}

\section{Discussion and Conclusion}\label{discussion-and-conclusion}

\begin{center}\rule{0.5\linewidth}{0.5pt}\end{center}

\begin{verbatim}

getwd()
setwd("/Users/tdain/OneDrive/Desktop/R Data/Borders and Public Opinion Paper")


a <- read.csv("public-opinion-paper/data/centroid_border_distances.csv")

b <- read.csv("C:/Users/tdain/OneDrive/Desktop/Stata Data/Graduate Work/Borders and Public Opinion Paper/data/province_key.csv")

install.packages("tinytex")

library(stringdist)
d <- expand.grid(a$District_Name,b$Name_1) # Distance matrix in long form
names(d) <- c("a_name","b_name")
d$dist <- stringdist(d$a_name,d$b_name, method="jw") # String edit distance (use your favorite function here)

library(fuzzyjoin); library(dplyr);

stringdist_join(a, b, 
                by = "name",
                mode = "left",
                ignore_case = FALSE, 
                method = "jw", 
                max_dist = 99, 
                distance_col = "dist") %>%
  group_by(name.x) %>%
  slice_min(order_by = dist, n = 1)

\end{verbatim}

\section{test}\label{test}

\begin{longtable}[]{@{}
  >{\raggedright\arraybackslash}p{(\columnwidth - 0\tabcolsep) * \real{0.0556}}@{}}
\toprule\noalign{}
\endhead
\bottomrule\noalign{}
\endlastfoot
\pagebreak \#\# Including Plots \\
\end{longtable}

\begin{center}\rule{0.5\linewidth}{0.5pt}\end{center}

\subsection{Citations}\label{citations}

\subsection{Other Common Things}\label{other-common-things}

\begin{quote}
This will create a block quote, if you want one.
\end{quote}

Dropping a footnote is easy.\footnote{See? Not hard at all.}




\newpage
\singlespacing 
\bibliography{\dummy{}}

\end{document}